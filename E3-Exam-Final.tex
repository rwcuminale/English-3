% Created 2018-06-01 Fri 11:35
% Intended LaTeX compiler: pdflatex
\documentclass[11pt]{article}
\usepackage[utf8]{inputenc}
\usepackage[T1]{fontenc}
\usepackage{graphicx}
\usepackage{grffile}
\usepackage{longtable}
\usepackage{wrapfig}
\usepackage{rotating}
\usepackage[normalem]{ulem}
\usepackage{amsmath}
\usepackage{textcomp}
\usepackage{amssymb}
\usepackage{capt-of}
\usepackage{hyperref}
\usepackage[margin=1.0in]{geometry}
\setlength\parindent{0pt}

\author{Richard Cuminale}
\date{Spring 2019}
\title{English 3 Final Exam Review}
\hypersetup{
 pdfauthor={Richard Cuminale},
 pdftitle={English 3 Final Exam Review},
 pdfkeywords={},
 pdfsubject={},
 pdfcreator={Emacs 26.1 (Org mode 9.1.9)}, 
 pdflang={English}}
\begin{document}

\maketitle

\section{Sentence Expansion Words and Techniques}
\label{sec:orgeb1a0b8}

Make sure that you are clear on the purpose and correct grammatical
use of each term or concept. You should be able to identify each in an
example sentence and create your own sentences. Your sentences should
also be punctuated correctly.
\begin{itemize}
\item Because
\item But
\item So
\item Appositives
\item However
\item Therefore
\item Yet
\item Independent and Dependent Clauses / Complex Sentences
\item Parallel Structure
\end{itemize}

\section{Writing Process}
\label{sec:org8a001c2}

Draw below the diagram that represents the writing process as it works
while you write:
\vspace{6cm}

\newpage
Draw below the diagram that represents how the writer's focus and
priorities change over time.
\vspace{6cm}

What is Exploratory Writing?
\vspace{2cm}

Be able to list examples of Exploratory Writing:
\begin{itemize}
\item Annotation
\item Free-writing
\item Summarizing
\item Clustering / Word Web
\item Outlining
\end{itemize}
\vspace{1cm}

What is Clarifying Writing?
\vspace{2cm}

Be able to list examples of Clarifying Writing:
\begin{itemize}
\item Introduction
\item Outlining
\item Organizing
\item Compressing
\item Proofreading
\end{itemize}

\newpage
\section{Narrative Analysis}
\label{sec:orge1ba52b}

On the final you must list and define the five narrative elements. 
\begin{enumerate}
\item Setting:
\item Character:
\item Conflict:
\item Plot:
\item Point of View (Storyteller):
\end{enumerate}

How do you discover the conflict in a narrative?
\vspace{3cm}

\section{Poetry Analysis}

On the final you must define the following 5 terms \emph{correctly}
and be prepared
to discover and explain the metaphor in a provided literary text.
They are not the only terms you should know upon moving up from
English 3, but they are the most important.

\begin{itemize}
    \item Imagery
    \item Metaphor
    \item Simile
    \item Stanza
    \item Symbol
\end{itemize}

\subsection{Analysis Practice}

Discover and describe below the metaphor in the attached poetry selections.

\newpage

\section{Literary Analysis Essay}
\label{sec:orgc3bb5ec}
\textbf{``By Any Other Name'' by Santha Rama Rau}

Read the narrative essay, ``By Any Other Name,'' by Santha Rama
Rau. Using the prompt below, respond in a focused, well organized
essay that has an introduction, conclusion, body paragraphs, and text
evidence.
\vspace{1em}

Essay Prompt: \textbf{Analyze the relationship between the two
  students, Santha and Premilla and the British.}
\vspace{1em}

Notes:
\begin{itemize}
\item You may want to consider literary elements such as imagery,
characterization, selection of detail, and structure in your
response.
\item You may use your annotated story and any other notes/organizers
prepared prior to the final exam.
\item Include an introduction that gives the context of the essay and
makes a clear thesis statement.
\item Be sure to make clear, consistent references to the story in
your analysis.
\end{itemize}

\emph{You may bring your annotated text, notes, and outline, but your
  essay must be written on the paper provided on the day of the final
  exam.}
\end{document}
